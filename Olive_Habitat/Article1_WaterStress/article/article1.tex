\documentclass[a4paper,12pt]{article}
\usepackage[utf8]{inputenc}
\usepackage{graphicx}
\usepackage{geometry}
\usepackage{hyperref}
\usepackage{natbib}
\usepackage{booktabs}
\usepackage{float}
\usepackage{listings}
\usepackage{color}

% Page layout
\geometry{margin=1in}

% Code highlighting settings
\definecolor{codegreen}{rgb}{0,0.6,0}
\definecolor{codegray}{rgb}{0.5,0.5,0.5}
\definecolor{codepurple}{rgb}{0.58,0,0.82}
\definecolor{backcolour}{rgb}{0.95,0.95,0.92}

\lstdefinestyle{mystyle}{
    backgroundcolor=\color{backcolour},   
    commentstyle=\color{codegreen},
    keywordstyle=\color{magenta},
    numberstyle=\tiny\color{codegray},
    stringstyle=\color{codepurple},
    basicstyle=\ttfamily\footnotesize,
    breakatwhitespace=false,         
    breaklines=true,                 
    captionpos=b,                    
    keepspaces=true,                 
    numbers=left,                    
    numbersep=5pt,                  
    showspaces=false,                
    showstringspaces=false,
    showtabs=false,                  
    tabsize=2
}
\lstset{style=mystyle}

\title{\textbf{AI-Based Early Detection of Climate-Induced Water Stress in Olive Orchards Using Google Earth Engine: A Case Study of Ghafsai, Morocco}}
\author{Aachaq et al.}
\date{\today}

\begin{document}

\maketitle

\begin{abstract}
The Mediterranean basin, particularly the Maghreb, is facing unprecedented aridity. This study investigates the use of Artificial Intelligence (AI) and multi-sensor remote sensing to detect water stress early in olive orchards (\textit{Olea europaea} L.) in Ghafsai, Morocco. Using Google Earth Engine (GEE), we processed terabytes of data from Sentinel-2, MODIS, CHIRPS, and SMAP satellites covering the 2019-2024 drought. We propose a framework using Random Forest (RF) for soil moisture downscaling and Long Short-Term Memory (LSTM) networks for forecasting vegetation health. Results show that while traditional NDVI lags in detecting stress in the 'Picholine Marocaine' cultivar, Short-Wave Infrared (SWIR) and Thermal Infrared (TIR) indices offer a lead time of up to three months. The study also identifies soil-dependent vulnerability gradients, aiding precision management aligned with Morocco's ``Generation Green 2020-2030'' strategy.
\end{abstract}

\section{Introduction}

\subsection{The Climatological and Hydrological Crisis in the Maghreb}
\begin{itemize}
    \item \textbf{Climate Hotspot:} North Africa is warming faster than the global average, with Morocco facing chronic water deficits \cite{frontiers2025}.
    \item \textbf{Severe Drought (2019-2024):} The country experienced its longest/most severe drought sequence, with 2023 being the driest year in 80 years (rainfall <100mm in many areas) \cite{frontiers2025}.
    \item \textbf{Water Crisis:} Dam filling rates dropped to 28.5\% in 2024 despite capacity expansion \cite{frontiers2025}.
    \item \textbf{Impact:} Agriculture (80\% of water use) is devastated, especially rainfed (Bour) areas like Taounate.
\end{itemize}

\subsection{The Olive Sector: Resilience Under Siege}
\begin{itemize}
    \item \textbf{Economic Pillar:} Olives cover 1.2 million ha in Morocco (65\% of fruit tree area) \cite{inra2025}.
    \item \textbf{'Picholine Marocaine':} The dominant cultivar is historically drought-resilient but now threatened by intensified farming and climate change \cite{mdpi_phenotyping}.
    \item \textbf{Intensification Issues:} Shift to high-density planting competes for dwindling groundwater \cite{mdpi_saiss}.
    \item \textbf{"Silent Stress":} Olives close stomata to save water, staying green (high NDVI) even when stressed. Visual signs come too late (leaf shedding), necessitating early detection \cite{rdi_olive}.
\end{itemize}

\subsection{The Paradigm of Remote Sensing and AI}
\begin{itemize}
    \item \textbf{Solution:} Earth Observation (EO) via Google Earth Engine (GEE) overcomes ground monitoring limitations \cite{gee_sentinel2}.
    \item \textbf{Multi-Sensor Approach:}
        \begin{itemize}
            \item \textbf{Optical (Sentinel-2):} Vegetation structure \cite{gee_sentinel2}.
            \item \textbf{SWIR (Sentinel-2/MODIS):} Canopy water content \cite{ndwi_researchgate}.
            \item \textbf{Thermal (MODIS):} Land Surface Temperature (LST) / stomatal conductance \cite{gee_modis_et}.
            \item \textbf{Microwave (SMAP):} Root zone soil moisture \cite{gee_smap}.
        \end{itemize}
    \item \textbf{AI Methods:}
        \begin{itemize}
            \item \textbf{Random Forest (RF):} For handling non-linear relationships in spectral data \cite{hess_rf_downscaling}.
            \item \textbf{LSTM (RNN):} For forecasting temporal drought trajectories \cite{lstm_drought}.
        \end{itemize}
\end{itemize}

\section{Study Area: The Ghafsai Circle}

\subsection{Geographical and Topographical Setting}
\begin{itemize}
    \item \textbf{Location:} Pre-Rif domain, Northern Morocco (between Saïss plain and Rif mountains).
    \item \textbf{Terrain:} Rugged hills and deep valleys, promoting rapid runoff.
\end{itemize}

\subsection{Geological and Pedological Context}
\begin{itemize}
    \item \textbf{Soils:}
        \begin{itemize}
            \item \textbf{Vertisols/Luvisols (Valleys):} High clay, good water holding but prone to cracking.
            \item \textbf{Lithosols (Slopes):} Shallow, skeletal, low water storage.
        \end{itemize}
    \item \textbf{Challenge:} Heavy clay soils cause waterlogging in winter and hardness in summer.
\end{itemize}

\subsection{Agro-Ecological Characteristics}
\begin{itemize}
    \item \textbf{Systems:} Mostly extensive 'Picholine' agroforestry (intercropped with cereals).
    \item \textbf{Phenology:} Warmer winters are disrupting chill hour accumulation, affecting flowering (April-May).
\end{itemize}

\section{Materials and Data Acquisition}

\subsection{Satellite Constellations}
\begin{itemize}
    \item \textbf{Sentinel-2 MSI:} 10m resolution. Key for resolving small orchards and calculating NDVI/NDWI.
    \item \textbf{MODIS:} High frequency (8-day). Used for Evapotranspiration (ET) and LST.
    \item \textbf{CHIRPS:} Gridded rainfall data (5.5km) for drought monitoring (SPI).
    \item \textbf{SMAP:} Root zone soil moisture (9km), crucial for deep-rooted olives.
\end{itemize}

\subsection{Auxiliary Data}
\begin{itemize}
    \item \textbf{Topography:} NASA SRTM DEM for slope/aspect.
    \item \textbf{Land Use:} Custom olive orchard mask.
\end{itemize}

\section{Methodology}

\subsection{Development of Water Stress Indices}
\begin{itemize}
    \item \textbf{Meteorological:} SPI (3 and 6 month) from CHIRPS \cite{unspider_spi}.
    \item \textbf{Agricultural:} Vegetation Health Index (VHI) combining VCI (Vegetation) and TCI (Temperature). Threshold: VHI < 40 = Stress \cite{noaa_vhi}.
    \item \textbf{Physiological:}
        \begin{itemize}
            \item \textbf{NDWI:} Leaf water content (NIR vs SWIR) \cite{ndwi_researchgate}.
            \item \textbf{ETDI:} Evapotranspiration Deficit Index from MODIS \cite{mdpi_etdi}.
        \end{itemize}
\end{itemize}

\subsection{Machine Learning: Spatial Downscaling (Random Forest)}
\begin{itemize}
    \item \textbf{Goal:} Downscale SMAP soil moisture (9km) to orchard scale (10m).
    \item \textbf{Inputs:} Sentinel-2 indices, LST, topography.
    \item \textbf{Output:} 10m Soil Moisture Map.
    \item \textbf{Interpretability:} SHAP values used to rank feature importance \cite{frontiers_shap_rf}.
\end{itemize}

\subsection{Deep Learning: Forecasting (LSTM)}
\begin{itemize}
    \item \textbf{Goal:} Forecast VHI 1-3 months ahead.
    \item \textbf{Model:} 2 stacked LSTM layers with Dropout.
    \item \textbf{Input:} 12-month sequence of Precip, Soil Moisture, VHI.
    \item \textbf{Testing:} Trained on 2017-2021, Tested on 2022-2024 (drought years).
\end{itemize}

\section{Results and Analysis}

\subsection{Meteorological Drought (2017-2024)}
\begin{itemize}
    \item \textbf{Timeline:} 2019 onset $\rightarrow$ 2022/2023 Extreme Drought (-56\% rainfall anomaly).
    \item \textbf{SPI:} Cumulative SPI-24 dropped below -3.0, indicating cessation of groundwater recharge.
\end{itemize}

\subsection{Vegetation Health Evolution (VHI)}
\begin{itemize}
    \item \textbf{2017-18:} Healthy (VHI > 60).
    \item \textbf{2019-20:} Warning phase (VHI 40-50). Lagged behind meteorological drought due to soil buffering.
    \item \textbf{2022-24:} Collapse (VHI < 35). 68\% of orchards severely stressed. South-facing slopes degraded first.
\end{itemize}

\subsection{Spectral Sensitivity: NDVI vs. NDWI vs. LST}
\begin{itemize}
    \item \textbf{NDVI:} Poor early warning ($R^2=0.45$). Lags stress because leaves stay green.
    \item \textbf{NDWI:} Good early warning ($R^2=0.72$). Lead time +1 month. Detects turgor loss.
    \item \textbf{LST:} Best early warning ($R^2=0.68$). Lead time +1.5 months. Detects stomatal closure (heating).
\end{itemize}

\subsection{AI Model Performance}
\begin{itemize}
    \item \textbf{Random Forest (Downscaling):} $R^2=0.82$. Captured micro-variability (e.g., "wet islands" in deep soils).
    \item \textbf{Feature Importance:} NDWI (28\%) and LST (22\%) were top drivers, validating physiological stress theory.
    \item \textbf{LSTM (Forecasting):}
        \begin{itemize}
            \item \textbf{1-Month:} Accurate (MAE 4.5).
            \item \textbf{3-Month:} Correctly predicted trend direction in 78\% of cases. Useful for seasonal planning.
        \end{itemize}
\end{itemize}

\section{Discussion}

\subsection{The "Silent Stress" Paradox}
\begin{itemize}
    \item 'Picholine' prioritizes survival over yield ("green but no olives").
    \item Stomatal closure happens early; visual signs happen late.
    \item \textbf{Takeaway:} Must monitor LST/NDWI, not just NDVI.
\end{itemize}

\subsection{Soil Influence}
\begin{itemize}
    \item \textbf{Clay (Vertisols):} Buffered early drought but cracked in later years, damaging roots.
    \item \textbf{Calcareous:} Highly sensitive to "flash droughts" (no buffer).
\end{itemize}

\subsection{Policy Implications (Generation Green)}
\begin{itemize}
    \item \textbf{Zoning:} Use AI maps to ban intensification on shallow soils/steep slopes.
    \item \textbf{Varieties:} Need for more xeric rootstocks beyond Picholine.
    \item \textbf{Advisory:} Use LSTM forecasts for early warning apps (triggering pruning/irrigation).
\end{itemize}

\section{Conclusion}
\begin{itemize}
    \item \textbf{Crisis:} 2019-2024 drought challenges olive viability in Ghafsai.
    \item \textbf{Method:} Multi-sensor fusion (Optical + Thermal + Microwave) is mandatory.
    \item \textbf{Findings:} SWIR/Thermal indices beat NDVI (1-3 month lead time).
    \item \textbf{Tools:} RF and LSTM effectively downscale data and forecast stress.
    \item \textbf{Future:} AI-driven management is a survival imperative for the Maghreb.
\end{itemize}

\bibliographystyle{plainnat}
\bibliography{references}

\end{document}
